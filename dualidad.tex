% Created 2019-05-17 vie 12:16
% Intended LaTeX compiler: pdflatex
\documentclass[presentation]{beamer}
\usepackage[utf8]{inputenc}
\usepackage[T1]{fontenc}
\usepackage{graphicx}
\usepackage{grffile}
\usepackage{longtable}
\usepackage{wrapfig}
\usepackage{rotating}
\usepackage[normalem]{ulem}
\usepackage{amsmath}
\usepackage{textcomp}
\usepackage{amssymb}
\usepackage{capt-of}
\usepackage{hyperref}
\usetheme{default}
\author{Verania Hernández}
\date{17 de Mayo de 2019}
\title{Dualidad en programación lineal}
\hypersetup{
 pdfauthor={Verania Hernández},
 pdftitle={Dualidad en programación lineal},
 pdfkeywords={},
 pdfsubject={},
 pdfcreator={Emacs 25.2.2 (Org mode 9.2.1)}, 
 pdflang={English}}
\begin{document}

\maketitle
\begin{frame}{Outline}
\tableofcontents
\end{frame}


\begin{frame}[label={sec:org974a816}]{Introducción}
La dualidad es una manera de asociar un cierto problema de
programación lineal a cada problema de programación lineal.
\end{frame}

\begin{frame}[label={sec:orge439a1a}]{Ejemplos}
Consideremos el siguiente problema

\begin{equation*}
 \begin{aligned}
 \text{Maximizar} \quad & 2x_{1}+3x_{2}\\
 \text{sujeto a} \quad &
   \begin{aligned}
    4x_{1}+8x_{2} &\leq 12\\
    2x_{1}+x_{2} &\leq 3\\
    3x_{1}+2x_{2} &\leq 4\\
    x_{1} &\geq  0\\
    x_{2} &\geq 0
   \end{aligned}
 \end{aligned}
 \end{equation*}
\begin{itemize}
\item Podemos observar que bajo las restriccione la funcion objetico es
menor que 12 ya que:
\begin{equation*}
2x_{1}+3x_{2}\leq 4x_{1}+8x_{2}\leq 12.
\end{equation*}
\item Siguiente paso\ldots{}
\item Y el siguiente
\end{itemize}
\end{frame}

\begin{frame}[label={sec:orgb3880af}]{Teoremas}
\end{frame}
\end{document}